\section{Заключение}

По результатам выполненной работы можно сделать следующие выводы:

\begin{enumerate}
\item {
Автором было проведено исследование формата базы данных Wikipedia
и реализован программный пакет, позволяющий получить доступ к~тексту
и отдельным элементам разметки отдельных страниц.
}
\item {
Был реализован модуль для~сохранения статей из~дампа Wikipedia
в СУБД~MySQL с~возможностью последующего к произвольным страницам
по их заголовку или идентификатору. 
}
\item {
Проведены исследование и условная классификация существующих 
алгоритмов определения меры семантической близости предложений. 
Были проанализированы и показаны достоинства и недостатки этих методов.
}
\item {
Один из рассмотренных алгоритмов был реализован и запущен для~набора 
пар предложений, схожесть которых была оценена людьми. 
Среднее значение модуля разности оценок людей и алгоритма составило $0.15$.
Также был создано консольное приложение, выполняющее поиск предложений 
в~определенной статье Wikipedia, выбирая наиболее близкие по смыслу
к~предложению, введенному пользователем.
}
\item {
В процессе реализации были изучены программные продукты для работы
с~естественными языками, такие как CoreNLP и WordNet.
} 
\end{enumerate}

В~ходе дальнейших исследований предполагается изучение более производительных БД, таких~как 
Lucene и MongoDB, которые лучше приспособлены для~поиска в~массивах текстов подобного размера.

Также выдвинуты предложения по улучшению реализованного алгоритма:
\begin{enumerate}
\item{
При оценке степени близости отдельных вершин дерева связей предложения учитывать 
их расстояние от~корня.
}
\item{
Использовать более эффективные алгоритмы определения изоморфности деревьев связей.
}
\item{
Модифицировать алгоритм, приспособив его для поиска предложения в тексте.
В этом случае искомая величина будет мерой вхождения одного предложения в~другое,
и в соответствии с этим нужно производить все вычисления.
}
\end{enumerate}


