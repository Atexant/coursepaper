\section*{Введение}
\addcontentsline{toc}{section}{\hspace{7mm}Введение}

Набор программных инструментов, решающих различные задачи в области
автоматизированной обработки информации, вошёл в повседневный круг
обихода большого числа людей. К ним относятся различные локальные
приложения, такие как настольные базы данных или системы электронного
документооборота, а также распределённые продукты, требующие
значительно больших ресурсов, чем предоставляет персональный
компьютер, как, например, поисковая система в Интернет. При этом
интеллектуальная составляющая в применяемых алгоритмах остаётся всё
ещё достаточно слабой, всю работу, связанную с восприятием и
интерпретацией информации, продолжает выполнять человек.

Сфера искусственного интеллекта (ИИ), несмотря на то, что развивается
уже на протяжении нескольких десятилетий, не может похвастаться полным
достижением поставленных перед исследователями целей. Восприятие
информации и принятие решений в таком виде, как это делает человек,
остаётся вне достижимых научных пределов. Тем не менее, интенсивное
развитие получили некоторые отдельные отрасли, ранее считавшиеся
разделами ИИ, как, например, Data Mining, которой посвящена настоящая
курсовая работа.
