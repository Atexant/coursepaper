\documentclass{beamer}

\usepackage[russian]{babel}
\usepackage[utf8]{inputenc}
\usepackage{cmap}
\usepackage{graphicx}
\usepackage{xspace}
\usepackage{psfrag}
\usepackage{amsfonts}
%%\usepackage{amssymb}

%%\newcommand{\MARK}[1]{{\bf {\it #1}}}
%%\newcommand{\CODE}[1]{{\ttfamily #1}}

\setbeamertemplate{footline}[frame number]
\usecolortheme{seahorse}
\beamertemplateshadingbackground{white}{blue!3}

\begin{document}
\begin{frame}
\begin{center}
Кирюшкина Валентина \\
\vspace{1cm}
{\Large Исследование алгоритмов извлечения знаний \\
с~использованием материалов \\
свободной энциклопедии Wikipedia}\\ 
\end{center}
\end{frame}

\begin{frame}
\frametitle{Data Mining и Text Mining}
Data Mining --- процесс нахождения полезных закономерностей (useful patterns)  в~большом наборе данных.\\
\vspace{1cm}
Text Mining --- процесс выделения полезных закономерностей 
в~больших массивах текста на~естественном языке. 
Text mining = Text Data Mining.
\end{frame}

\begin{frame}
\frametitle{Задача извлечения знаний из Wikipedia}

Основная задача --- научиться извлекать как можно больше знаний на~основе Wikipedia,
которые могут быть использованы  при~создании различных прикладных интеллектуальных утилит.\\
\vspace{0.3cm}
Важные свойства Wikipedia:

\begin{enumerate}

\item {Одна статья --- одно понятие (сущность).}
\item{Предоставляет материалы по именам собственным.}

\end{enumerate}

Обработка имён собственных в~тексте может проводиться только при~наличии базы данных.

\end{frame}

\begin{frame}
\frametitle{Достоинства Wikipedia}

Важные достоинства:

\begin{itemize}
\item{обширность материалов;}
\item{регулярные обновления для~поддержания актуальности;}
\item{большое количество ссылок между статьями.}
\end{itemize}

\end{frame}

\begin{frame}
\frametitle{Основные направления Text Mining}
\begin{enumerate}
\item{Классификация.}
\item{Кластеризация.}
\item{Извлечение информации (эвристические алгоритмы поиска закономерностей).}
\item{Построение графа знаний в~виде семантической сети.}

\end{enumerate}
\end{frame}

\begin{frame}
\frametitle{Классификация документов}
$$D \mapsto L, L \in \mathbb{L},$$

где $\mathbb{L}$ --- множество классов документов.

\vspace{1cm}

В~отличие от~кластеризации множество классов задано изначально.
\end{frame}

\begin{frame}
\frametitle{Кластеризация документов}
$$D \mapsto P, P \in \mathbb{P},$$
где $\mathbb{P}$ --- множество кластеров.

\vspace{1cm}

Документы разделяются на~подмножества таким образом, что:

\begin{itemize}
\item{внутри одного подмножества документы \underline{однородны};}
\item{документы из разных подмножеств \underline{разнородны}.}
\end{itemize}

$Dist(d_{1},d_{2})$ --- мера расстояния между документами.
\end{frame}

\begin {frame}
\frametitle{Построение тезауруса}
\textbf{Задача}: Построение тезауруса - словаря, отражающего семантические отношения
между словами, в данном случае --- близость между словами (синонимичность).\\
\vspace{1cm}
\textbf{Входной материал}: Множество статей, которые рассматриваются как понятия.\\
\vspace{1cm}
\textbf{Выходной материал}: значение функции, аргументами которой являются две~статьи, 
соответствующие понятиям, степень близости между которыми необходимо установить.\\
\end{frame}

\begin{frame}
\frametitle{Алгоритм построения тезауруса}
Поскольку Wikipedia состоит из~множества статей (понятий) и гиперссылок между ними, она может быть представлена
в виде графа $G = \{V,E\}$, где $V$  --- это множество статей, а $E$ ---это множество ссылок.\\
\vspace{0.3cm}
Близость между понятиями(статьями) $v_i$ и $v_j$ зависит от~следующих параметров:

\begin{itemize}
\item{количества путей между $v_i$ и $v_j$;}
\item{длины каждого пути между $v_i$ и $v_j$;}
\item{количество входящих ссылок.}
\end{itemize}

Путь между двумя статьями --- это последовательность ссылок (их~количество определяет длину пути), которая соединяет данные статьи.

\end{frame}

\begin{frame}
\frametitle{Извлечение информации}
Задача извлечения информации заключается в~получении структурированных знаний
из~текста на~естественном языке.\\

\vspace{0.5cm}
Подзадачи:
\begin{enumerate}
\item{Распознавание именованных сущностей.}
\item{Извлечение информации из~слабоструктурированных текстов.}
\item{Нахождение терминологии, специфичной для данного документа.}
\end{enumerate}

Данные могут быть представлены в~виде объектов с атрибутами.

\end{frame}

\begin{frame}
\frametitle{Пример}
\textbf{Пример.} Дано предложение на~естественном языке:\\ 
``Robert L. James, chairman and chief executive
officer of McCann-Erickson, is going to retire on July 1st. He will be replaced
by John J.Donner, Jr., the agencies chief operating officer.''\\
\vspace{1cm}
Получим следующие атрибуты: организация (\textsl{McCann-Erickson}), должность (\textsl{chief~executive~officer}), дата(\textsl{July~1}), имя уходящего с~должности человека (\textsl{Robert~L.~James}), имя
приходящего на~должность человека (\textsl{John~ J.~Donner,~Jr.}).
\end{frame}

\begin{frame}
\frametitle{Weka (Waikato Environment for Knowledge Analysis)}

Weka --- свободно распространяемый пакет для анализа данных,
написанный на языке Java в университете Уайкато (Новая Зеландия). \\
\vspace{0.3cm}
Группы алгоритмов Weka:
\begin{itemize}
\item{классификация;}
\item{кластеризация;}
\item{задача выборки атрибутов;}
\item{задача извлечения правил ассоциаций.}
\end{itemize}

\end{frame}

\begin{frame}
\frametitle{Объекты и атрибуты}
Нередко данные  об объекте хранятся в виде фиксированного списка атрибутов.

\vspace{0.5cm}

Опишем базу данных, которую будем использовать на следующих слайдах в~качестве примера.
Возьмем в качестве рассматриваемых объектов разных животных.
Список атрибутов следующий:
вид(\textsl{строка}), средний вес(\textsl{число}), кормление молоком(\textsl{булевый}), наличие шерсти(\textsl{булевый}), длина названия вида(\textsl{число}).
\end{frame}

\begin{frame}
\frametitle{Извлечение правил ассоциаций}

Основной задачей является поиск интересных и полезных закономерностей в базе данных.\\
\vspace{0.5cm}
{\bf Пример.}
Выберем два атрибута: средний вес и наличие шерсти.
Можем получить следующее логическое выражение: 
если средний вес = 50 кг и шерсть = 1  $\to$ вид = волк или вид = шимпанзе.
И так далее.

\end{frame}

\begin{frame}
\frametitle{Выборка атрибутов}
Задача сортировки атрибутов объектов по~их~информационной содержательности.\\

\vspace{0.5cm}
{\bf Пример.}
В нашей базе данных в результате действия алгоритмов выборки атрибутов получим отсортированный список: вид, средний вес, кормление молоком, наличие шерсти.\\
Очевидно, что атрибут «длина названия вида» не несет никакой информационной ценности.
\end{frame}

\begin{frame}
\frametitle{Критика идеи построения тезауруса и проекта Weka}

Сложности при работе с Weka:
\begin{enumerate}
\item{Необходимость построения базы знаний  с объектами и атрибутами: на~данном этапе качественно почти невозможно.}
\item{Большое число атрибутов для~обработки: вычислительно трудоёмко.}
\end{enumerate}
\vspace{0.7cm}
Алгоритм построения тезауруса основан на классификации и кластеризации.
Сами по себе алгоритмы классификации и кластеризации являются эвристическими с~крайне ограниченным потенциалом для~дальнейших исследований.

\end{frame}

\begin{frame}
\frametitle{Граф знаний}
Граф знаний есть ориентированный(в большинстве случаев) граф, в котором вершины представляют так называемые концепты (токены и типы),
а ребра --- отношения между ними.
Причем отношения могут быть как бинарные, так и многомерные.\\ 
\vspace{0.5cm}
\textsl{Токену} соответствует какой-либо реальный объект или абстрактное понятие.
 На~графе он обозначается как $\Box$. \\
\vspace{0.5cm}
Токен, отражающий некоторое общее понятие, в терминологии графа знаний называется \textsl{типом}.\\
\end{frame}

\begin{frame}
\frametitle{Оценка близости понятий на графе}

Известны некоторые готовые варианты графов, узлами которых являются понятия.
Примером может служить Wordnet, который содержит отношения для~обозначения гиперонимов 
между синонимичными рядами --- синсетами.\\
\vspace{0.3cm}
Реализован алгоритм, который определяет близость понятий на~графе гиперонимов на~основе формулы:

$$ Sim(w_1, w_2) = \frac{ 2\times d(l, r) }
	               { d(w_1, l) + d(w_2, l) + 2\times d(l,r), }$$

где $w_1,w_2$ --- рассматриваемые концепты, 
$l$ --- их~наименьший общий предок в~дереве гиперонимов,
$r$ --- базовое понятие --- корень дерева, 
а $d(n_1,n_2)$ -- 
кратчайшее расстояние между вершинами $n_1$ и $n_2$.

\end{frame}

\begin{frame}
\frametitle{Заключение}

\begin{itemize}

\item{Проведено теоретическое исследование алгоритмов извлечения знаний и применение их для работы с Wikipedia:

\begin{itemize}
\item{Изучены алгоритмы классификации и кластеризации;}
\item{Исследован алгоритм построения тезауруса на основе Wikipedia;}
\item{Произведено знакомство с библиотекой алгоритмов машинного обучения для решения задач интеллектуального анализа данных Weka;}
\item{Изучены теоретические основы построения графа знаний.}
\end{itemize}}
\item{Изучен программный пакет для работы с текстами на~естественном языке WordNet.}
\item{Реализован алгоритм нахождения близости между двумя словами.}

\end{itemize}
\end{frame}

\begin{frame}
\frametitle{Спасибо за внимание!}
\begin{enumerate}

\item{Zhang, L.  Knowledge graph theory and structural parsing // Twente University Press. – 2002. -  P. 216}
\item{Dingjia L. A Dependency Grammar and WordNet Based Sentence Similarity Measure // Journal of Computational Information Systems. – 2012. – Vol. 8, № 3. — P. 1027-1035.}
\item{Hall M. The WEKA data mining software: an update / M. Hall, E. Frank, G. Holmes, B. Pfahringer, P. Reutemann, I.an H. Witten // ACM SIGKDD Explorations Newsletter. – 2009. – Vol. 11, Issue 1.}
\item{Hotho A. A Brief Survey of Text Mining / A. Hotho, A. Nürnberger, G. Paass //  LDV Forum – GLDV Journal for Computational Linguistics and Language Technology. – 2005. – Vol. 20, №1 - P. 19-62.}
\item{Nakayama K. Wikipedia Mining for an Association Web Thesaurus Construction / K. Nakayama, T. Hara , S. Nishio // WISE. – 2007. – Springer. – P.322-334.}
\end{enumerate}
\end{frame}

\end{document}
