В рамках данной научной работы был реализован алгоритм определения близости слов
на основании их положения в дереве понятий, представленном в семантической сети
английского языка Wordnet\cite{wordnet}.
Были проведены запуски на парах слов, упомянутых в \cite{complexSim},
для которых уже известна средняя оценка близости, данная людьми.

Wordnet\cite{wordnet} - лексическая база данных английского языка,
разработанная
в Принстонском Университете, и выпущенная вместе с сопутствующим
программным обеспечением.
Основной единицей Wordnet является не отдельное слово, а так
называемый синсет (синонимический ряд),
объединяющий в себе множество слов, имеющих схожий смысл, выражающих
одно и то же понятие.
Кроме набора синсетов в базе представлены также и семантические
отношения между ними,
которые позволяют определить понятия-антонимы рассматриваемого концепта,
а так же его гиперонимы - более общие понятия, отражающие надмножество
к исходному.
Например для понятия "собака" гиперонимами могут являться такие концепты
как "домашнее животное" или "семейство собачьи".
Очевидно, что на основе отношений-гиперонимов можно построить
ациклический ориентированный граф - дерево, вершинами которого
будут отдельные понятия-синсеты, а направление дуг указывает на
гиперонимы к ним.
Кроме того следует упомянуть о существовании базового концепта, являющегося
гиперонимом ко всем остальным синсетам. На графе он будет корнем дерева,
из которого не выходит ни одна дуга, и до которого существует путь от
каждой вершины.

Автором был реализован алгоритм, впервые описанный  в статье
\cite{inproceedings},
и рекомендуемый к реализации в \cite{complexSim}.
Для вычисления величины, определяющей близость пары понятий, предлагается
воспользоваться следующей формулой:

\begin{equation}
Sim(w1, w2) = \frac{ 2\times d(l, r) }
                               { d(w1, l) + d(w2, l) + 2\times d(l,r) }
\end{equation}
где w1,w2 - рассматриваемые концепты, l - их наименьший общий предок в
дереве гиперонимов,
r - базовое понятие - корень дерева, а d(n1,n2) - кратчайшее
расстояние между вершинами n1 и n2.

Как видно из построения, полученная величина нормирована - ее значения лежат
в интервале [0,1].
Причем ее значения зависят от следующих факторов:
1. Как далеко в дереве расположены друг от друга сравниваемые концепты .
 (Чем ближе расположены, тем больше величина близости)
2. Насколько далеко от корня расположен наименьший общий предок.
  (Чем дальше расположен, тем больше значение)
Расстояние от корня до понятия по сути определяет его специфичность,
чем больше расстояние, тем уже рассматриваемый концепт.
Таким образом, даже если расстояние между сравниваемыми вершинами
относительно велико, но рассматриваемые понятия принадлежат
достаточно узкому классу, т.е. их общий предок расположен далеко от корня,
в связи с  этим вычисленная таким образом мера близости будет
сравнительно велика.

\subsection{Особенности реализации}
В качестве языка программирования для реализации был выбран Java,
а для взаимодействия с базой Wordnet использовалась библиотека JWNL\cite{jwnl}.
Эта библиотека дает удобные инструменты для работы с отношениями на синсетах,
и такие задачи как нахождение общего предка, расстояния между ними не
вызвало затруднений.

Однако при запуске на упомянутом выше тестовом наборе было замечено
завышение оценок
по сравнению со средними оценками людей.
Этого удалось избежать, снизив коэффициент "2" перед членом формулы d(l,r)
до значения 0,8. Вероятно коэффициент "2" был рассчитан на
семантические сети меньшего
размера, чем Wordnet, где длина пути от корня до достаточно общих
понятий сравнительно велика.

Кроме того у  автора возникли сложности, связанные с многозначностью
слов английского языка.
Даже при верном определении части речи слова возможна ситуация, когда
оно может присутствовать
в разных синсетах, причем в зависимости от того, какой из них будет выбран,
итоговое значение меры близости может иметь разброс от близкого к нулю
до единицы.
Например в случае сравнения глаголов "to go" и "to move",  первый
можно перевести как "уходить",
и как "начать что-либо", и в зависимости, от выбранного смысла, они
могут быть как очень похожи,
так и иметь совершенно разный смысл.
В таких случаях верное значение слов можно получить только из
контекста, но это выходит
за рамки решаемой задачи: на вход алгоритму подается только само слова
и его часть речи.
Другим решением этой проблемы может стать некоторого статистичского
среднего значения близости из выборки, составленной из значений меры
близости для всех пар синсетов,
а также был рассмотрен вариант выбрать те синсеты, при которых
эта мера достигает своего максимума.
В соответствии с паттерном проектирования Стратегия, был выделен
отдельный интерфейс
AbstractSelectionEvaluationStategy для определения итогового значения
меры близости, используя
в качестве входных данных массив значений меры для всех пар синсетов.
Были созданы несколько его реализаций, которые выбирают:
1. Среднее арифметическое выборки
2. Максимальное значение из всей выборки
3. Статистическую медиану выборки (Может использоваться для
нормального распределения)

В качестве базовой было зафиксировано использование реализации с
максимальным значением,
но предполагается, что в дальнейшем выбор стратегии будет усложнен.

\subsection{Результаты}
Как уже было сказано выше, были произведены замеры для 97 пар слов,
с данными для них оценками семантической близости. Результаты в виде таблицы
представлены в приложении к работе.
Среднее значение отклонения оценок людей и алгоритма составляет 0.150776,
что можно считать удовлетворительным результатом.
Наибольшее отклонение - 0.55947 - было достигнуть для пары слов
"fruit" и "food",
где оценка данная людьми была равна 0.77, а значение, предложенное алгоритмом,
составило 0.210526. Заниженная в данном случае оценка связана с несовершенством
базы Wordnet, где понятию "fruit" соответствует только значение "плод растения",
когда с точки зрения большинства людей наиболее важным свойством фрукта
является возможность употребления его в пищу. То есть для многих людей
близким гиперонимом понятия "фрукт" является "еда", о чем и говорит оценка 0.77.


%%Дальше библиография
----

strategy
http://en.wikipedia.org/wiki/Strategy_pattern

jwnl
http://sourceforge.net/projects/jwordnet/
complexSim
Dingjia LIU, Zequan LIU, Qian DONG, A Dependency Grammar and WordNet
Based Sentence Similarity Measure. Journal of Computational
Information Systems 8: 3 (2012) 1027-1035

wordnet = http://wordnet.princeton.edu/

@inproceedings{Wu:1994:VSL:981732.981751,
 author = {Wu, Zhibiao and Palmer, Martha},
 title = {Verbs semantics and lexical selection},
 booktitle = {Proceedings of the 32nd annual meeting on Association
for Computational Linguistics},
 series = {ACL '94},
 year = {1994},
 location = {Las Cruces, New Mexico},
 pages = {133--138},
 numpages = {6},
 url = {http://dx.doi.org/10.3115/981732.981751},
 doi = {10.3115/981732.981751},
 acmid = {981751},
 publisher = {Association for Computational Linguistics},
 address = {Stroudsburg, PA, USA},
}
